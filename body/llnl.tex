% Opener
I am honored to submit an application to join Lawrence Livermore National Laboratory. (LLNL)
Currently, I'm finishing my dissertation for the PhD program in Stony Brook University (SBU) Department of Physics and Astronomy, and conducting research in the fields of Physics and Computational Neuroscience in Cold Spring Harbor Laboratory (CSHL) as a part of my studies.
In my academic career, my main interest is dynamical systems and machine learning.
My vocational pursuit is application of the techniques that I have learned during my education, and development of new methods, in climate change mitigation.
I believe that my interest and education so far makes me very well suited for the position of "Modeling, Simulation, and Optimization for Energy and Climate systems" (REF2630P).
I am very happy to see this listing, the topics are exactly the type of scientific contribution I would like to provide.

During my graduate school, I have worked on several projects relating to the use of machine learning methods.
My main goal in combining Physics and Neuroscience has been the use of physics analysis methods to explain the computations performed by the brain.
My research goals have been directed towards Neuroscience, which has mainly been modeling neural computation of biological circuitry using artificial neural networks.
I have written my own numerical simulations to solve continuous network models in this end.
One specific project related to figuring out how animals navigate to odors, I worked on fluid mechanics.
My dissertation contains mostly my other contributions due to the implementation of this project taking too much time compared to the projects that I received funding for.
However, working with fluid mechanics has been very interesting, even though I have been unable to explore as much as I liked.
The method I am mostly interested in is mesh-free methods, specifically Lagrangian formulations such as smoothed-particle hydrodynamics.~(SPH)
For my simulations that involved numerical solutions to differential equations, I mainly used MATLAB.
I have experience using OpenFOAM to solve Navier-Stokes equation on simple geometries.
I also have experience with using PyTorch due to the collaborative environment in my current laboratory, and a lot of my colleagues projects involve deep neural networks.

In terms of topic interests, I believe that climate change is the biggest threat to humanity and I feel driven to contribute to climate change research and mitigation.
I have been looking to apply the theoretical physics I learned during my undergraduate and masters stage of my academic career to improve climate simulations.
One point that I bring is that I consider machine learning to be very useful in combating climate change.
There are advances in utilizing machine learning in industrial design for optimal engineering, which can transfer very well to climate change mitigation technologies.
(Such as carbon capture, fuel cell design, etc.)

I believe with my qualifications, education and scientific interest, this position is a great match for me.
I identify some qualifications that I am lacking in.
One is that I do not have experience simulating flow through porous materials or have worked with massively parallelized computing before.
I found that my theoretical physics knowledge translates well when learning adjacent fields such as material science.
Also, my computational experience has not been on massively parallel systems so far.
I do have some proficiency with parallelization within a single machine, and I am proficient in administrating and scripting in Linux systems.
I believe I can make up for this lack quickly with my knowledge.

I hope my application is received well, and I am very eager to discuss the potential to conduct research in this field.
Looking forward to hearing from you.
