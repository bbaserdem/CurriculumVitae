% Opener
Good evening.

% Beginning statements
I am working as a postdoc at Cold Spring Harbor Laboratory (CSHL) with Dr. Alexei Koulakov,
continuing the work we did together in my Ph.D. in Physics and Neuroscience.

I started my journey as a Ph.D. candidate in Stony Brook University (SBU) Department of Physics and Astronomy.
My main interest throughout my studies has been mathematical modeling of complex dynamical systems.
This is what drew me into research at the intersection of physics and neuroscience at CSHL as part of my doctoral studies.
I focused on interdisciplinary modeling of systems that had not been modeled before,
using tools from machine learning (ML), dynamical computational graphs, and fluid dynamics.

My first project was to determine if self-assembly of a genetic network is mathematically plausible (it is!),
and to develop and optimize the simulation.~\cite{connclone}
I had to optimize on a low level how interactions would work and generate my own time series data.
On top of this generated data, I ran statistical analysis on the simulation.
This whole process was collaborative, where we had to communicate both low level
and high level details of the project to various stakeholders and colleagues.
In another project, I used fluid mechanics to model the dispersion of smell in the air,
and used reinforcement learning (RL) to model how an animal would learn to navigate to the source of the smell.
Besides working on my own projects, I held and participated many discussion groups
on developments in ML, and held workshops on how to implement them.

In my future work, I would like to leverage the technical skills I developed during my Ph.D.
and to continue developing new ones to model complex systems.
I’m excited to continue integrating mathematical modeling with AI toolkits.
I’m curious to learn more about modeling power,
both its generation and distribution through the grid because of its complexity,
constraints, and my humanist interest in the future of power development.

%I conducted my Ph.D. research in Physics and Neuroscience with Dr. Koulakov,
%and stayed with him post graduation to conclude the research we have done together.
%I did my Ph.D. in Stony Brook University Department of Physics and Astronomy.

% Main payload
%My main interest in pursuing my degree has been mathematical modelling of complex dynamic systems.
%This is the reason why I chose to conduct research at the intersection of physics and neuroscience at CSHL.
%During my research, I worked on several projects that involved mathematical modelling of systems that were not modelled before.
%My first research project was to figure out if self-assembly of a genetic network is mathematically plausible, and to write and optimize the simulation.
%This was a hypothetical system, however was a great project for me.
%I had to optimize on a low level on how interactions would work, and generate my own time series data.
%On top of this generated data, I ran statistical analysis on the simulation.
%This whole process was collaborative with my colleages, where we had to communicate both low level and high level details of the project.
%While working on projects, our lab also focused on learning about machine learning (ML) in general,
%and we held many meetings where we discussed and presented recent developments in ML.
%One of the projects I worked on for my dissertation, I used fluid mechanics to model the dispersion of smell and reinforcement learning (RL) to model how an animal would learn to navigate to the source of the smell.

%During my Ph.D. work, I mostly worked on producing mathematical models to either analyse or simulate complicated systems.
%In my further work, I would like to both leverage the technical skills I developed during my Ph.D. and to gain new ones to model different systems.
%Interaction of complex and dynamical computational graphs has always been my main interest that drove my research direction, and I believe my skillset is pretty well suited for working with analysing the power market.
%I'm really interested in integrating mathematical models with artificial intelligence (AI) toolkits such as ML, RL.
